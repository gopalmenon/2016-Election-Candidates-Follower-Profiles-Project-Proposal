% This is "sig-alternate.tex" V2.1 April 2013
% This file should be compiled with V2.5 of "sig-alternate.cls" May 2012
%
% This example file demonstrates the use of the 'sig-alternate.cls'
% V2.5 LaTeX2e document class file. It is for those submitting
% articles to ACM Conference Proceedings WHO DO NOT WISH TO
% STRICTLY ADHERE TO THE SIGS (PUBS-BOARD-ENDORSED) STYLE.
% The 'sig-alternate.cls' file will produce a similar-looking,
% albeit, 'tighter' paper resulting in, invariably, fewer pages.
%
% ----------------------------------------------------------------------------------------------------------------
% This .tex file (and associated .cls V2.5) produces:
%       1) The Permission Statement
%       2) The Conference (location) Info information
%       3) The Copyright Line with ACM data
%       4) NO page numbers
%
% as against the acm_proc_article-sp.cls file which
% DOES NOT produce 1) thru' 3) above.
%
% Using 'sig-alternate.cls' you have control, however, from within
% the source .tex file, over both the CopyrightYear
% (defaulted to 200X) and the ACM Copyright Data
% (defaulted to X-XXXXX-XX-X/XX/XX).
% e.g.
% \CopyrightYear{2007} will cause 2007 to appear in the copyright line.
% \crdata{0-12345-67-8/90/12} will cause 0-12345-67-8/90/12 to appear in the copyright line.
%
% ---------------------------------------------------------------------------------------------------------------
% This .tex source is an example which *does* use
% the .bib file (from which the .bbl file % is produced).
% REMEMBER HOWEVER: After having produced the .bbl file,
% and prior to final submission, you *NEED* to 'insert'
% your .bbl file into your source .tex file so as to provide
% ONE 'self-contained' source file.
%
% ================= IF YOU HAVE QUESTIONS =======================
% Questions regarding the SIGS styles, SIGS policies and
% procedures, Conferences etc. should be sent to
% Adrienne Griscti (griscti@acm.org)
%
% Technical questions _only_ to
% Gerald Murray (murray@hq.acm.org)
% ===============================================================
%
% For tracking purposes - this is V2.0 - May 2012

\documentclass{sig-alternate-05-2015}
\usepackage{listings}
\usepackage{float}

\begin{document}

% Copyright
%\setcopyright{acmcopyright}
%\setcopyright{acmlicensed}
%\setcopyright{rightsretained}
%\setcopyright{usgov}
%\setcopyright{usgovmixed}
%\setcopyright{cagov}
%\setcopyright{cagovmixed}

%Conference
%\conferenceinfo{PLDI '13}{June 16--19, 2013, Seattle, WA, USA}

%\acmPrice{\$15.00}

%
% --- Author Metadata here ---
%\conferenceinfo{WOODSTOCK}{'97 El Paso, Texas USA}
%\CopyrightYear{2007} % Allows default copyright year (20XX) to be over-ridden - IF NEED BE.
%\crdata{0-12345-67-8/90/01}  % Allows default copyright data (0-89791-88-6/97/05) to be over-ridden - IF NEED BE.
% --- End of Author Metadata ---

\title{Project Proposal}
\subtitle{CS 7930 Social Media Mining\\
Spring 2016}
%\subtitle{[Extended Abstract]
%\titlenote{A full version of this paper is available as
%\textit{Author's Guide to Preparing ACM SIG Proceedings Using
%\LaTeX$2_\epsilon$\ and BibTeX} at
%\texttt{www.acm.org/eaddress.htm}}}
%
% You need the command \numberofauthors to handle the 'placement
% and alignment' of the authors beneath the title.
%
% For aesthetic reasons, we recommend 'three authors at a time'
% i.e. three 'name/affiliation blocks' be placed beneath the title.
%
% NOTE: You are NOT restricted in how many 'rows' of
% "name/affiliations" may appear. We just ask that you restrict
% the number of 'columns' to three.
%
% Because of the available 'opening page real-estate'
% we ask you to refrain from putting more than six authors
% (two rows with three columns) beneath the article title.
% More than six makes the first-page appear very cluttered indeed.
%
% Use the \alignauthor commands to handle the names
% and affiliations for an 'aesthetic maximum' of six authors.
% Add names, affiliations, addresses for
% the seventh etc. author(s) as the argument for the
% \additionalauthors command.
% These 'additional authors' will be output/set for you
% without further effort on your part as the last section in
% the body of your article BEFORE References or any Appendices.

\numberofauthors{1} %  in this sample file, there are a *total*
% of EIGHT authors. SIX appear on the 'first-page' (for formatting
% reasons) and the remaining two appear in the \additionalauthors section.
%
\author{
% You can go ahead and credit any number of authors here,
% e.g. one 'row of three' or two rows (consisting of one row of three
% and a second row of one, two or three).
%
% The command \alignauthor (no curly braces needed) should
% precede each author name, affiliation/snail-mail address and
% e-mail address. Additionally, tag each line of
% affiliation/address with \affaddr, and tag the
% e-mail address with \email.
%
% 1st. author
\alignauthor
Gopal Menon\\
       \affaddr{Computer Science Department}\\
       \affaddr{Utah State University}\\
       \affaddr{Logan, UT}\\
       \email{gopal.menon@aggiemail.usu.edu}
}
% There's nothing stopping you putting the seventh, eighth, etc.
% author on the opening page (as the 'third row') but we ask,
% for aesthetic reasons that you place these 'additional authors'
% in the \additional authors block, viz.

% Just remember to make sure that the TOTAL number of authors
% is the number that will appear on the first page PLUS the
% number that will appear in the \additionalauthors section.


\maketitle


%
% The code below should be generated by the tool at
% http://dl.acm.org/ccs.cfm
% Please copy and paste the code instead of the example below. 
%



%
% End generated code
%

%
%  Use this command to print the description
%

% We no longer use \terms command
%\terms{Theory}

\section{Background}
The current front runners in the 2016 Presidential Election are Donald Trump from the Republican party and Hillary Clinton from the Democratic party. 

The Republican party base has been steadily moving away from traditional candidates as they reportedly feel that they do not have a say in matters that are of importance to them. The rise of the Tea Party and the influence of Talk Radio, has reportedly made the Republican base disillusioned with Congress. In the last few elections, traditionally safe Republican candidates have been unseated in the primaries by Tea Party candidates.There has been growing discontent with President Obama's perceived Liberal agenda and people in the Republican base seem to want to start afresh with new candidates as they seem to have lost trust in career politicians.

Trump is not a career politician and has no experience in government. He started out as an outside candidate who was not taken seriously. However he has succeeded in becoming the front-runner by appealing to the base of the republican party. He has been accused by many of being racist, fear mongering and wrong on things that he has put forward as facts. He is accused of being neither religious nor conservative and has still emerged as the likely nominee from his party. He is reportedly disliked by members of his own party, but despite this, has emerged as the leading candidate.

The Democratic party traditionally has its strong base in the East and West coasts, Liberals and minority communities. Unlike the Republican party, the Democrats have not been going through any internal turmoils. However, their candidate has had her share of problems. Clinton has been associated with scandals from the past like her husband's infidelities and the Whitewater investigations. During her term as Secretary of State, she was held responsible for the death of the American ambassador along with three other persons, during the attach on the consulate in Bengazhi. After her term, she was faced with the email scandal. Outside of her party she is reportedly disliked and is accused of being corrupt and is a very divisive figure.

The Democrats, unlike the Republicans, started out with just four candidates and were left with only candidates - Hillary Clinton and Bernie Sanders. At the time of writing this proposal, Clinton seems to be the likely nominee from the party.

\section{Research Focus}

My research focus will be on using Tweets to find the profile of the supporters of the two candidates. I am primarily interested in finding out what issues matter to the supporters. 

\subsection{Why is it interesting}
I find this interesting because of the success of the Trump campaign. He had been written off initially by news analysts and they did not expect him to last beyond the initial phase. His continued success has taken many people by surprise. By finding the issues that matter to his supporters, I can find out what are the key factors that drive the apparent demand for change among the base. Since the career politicians have not been successful, I can find out how the Republican party is out of touch with the issues that matter most to its base. The reasons could be perceived loss of employment opportunities due to globalization and illegal immigration, changes in the demographics of the voter population, perceived Liberal policies of President Obama, opposition to gay marriage and the Affordable Care Act (also known as Obamacare), fear of terrorism or it could be something that I have not thought of. Whatever the reasons are, they will be interesting as the emergence of Trump is an unforeseen phenomenon that maybe even he does not understand.

The identification of issues important to the supporters of Clinton will be a bonus, but may not be as interesting as the issues that motivate Trump supporters.

\subsection{Who will be interested}
Social scientists, people who follow politics and key members of the two parties will potentially be interested in knowing the results of the investigation. 

The Republican party will be interested as they apparently need to change their focus since all their traditional candidates have been rejected by the primary voters. After the election of President Obama in 2008, many news analysts said that the Republican party needs to change its attitude towards that Latinos, who consist of the largest minority, in order to be viable in the future. The party may need to change in order to survive.  

The Democrats on the other hand, have not had any issues with their traditional base. They would be interested in knowing what issues are important to their opponents and would want to use this information in order to to better compete.

\section{Previous Research}

Previous research has been done by Kloumann and Kleinberg \cite{SeedSet} on identifying members of a community starting from a small seed set. The seed set was expanded using PageRank and other algorithms. The PageRank algorithm reportedly had good results in identifying key members of a community given the seed set.
\section{Approach}
I plan to use the tweets collected as part of Assignment 1 as the seed set. If required, I will retrieve more tweets to use as part of the seed set. The PageRank algorithm will be used to identify important members of the community by sorting the community by PageRank and selecting the ones with the higher values. Once the important community members are available, I plan to use Latent Dirichlet allocation (LDA) topic modeling on their tweets using the R package for finding the topics important to them.

\subsection{Design}
\subsection{Tools and Techniques}
\section{Milestones}
\section{Evaluation Criteria for Results}

\begin{thebibliography}{1}

\bibitem{SeedSet}
Kloumann, I.M. and Kleinberg, J.M., 2014, August. Community membership identification from small seed sets. In \emph{Proceedings of the 20th ACM SIGKDD international conference on Knowledge discovery and data mining} (pp. 1366-1375). ACM.

\bibitem{IRB}
Manning, C. D., Raghavan, P., \& Sch\"utze, H. (2008). Introduction to information retrieval/Christopher D.

\end{thebibliography}
%
% The following two commands are all you need in the
% initial runs of your .tex file to
% produce the bibliography for the citations in your paper.
%\bibliographystyle{abbrv}
%\bibliography{sigproc}  % sigproc.bib is the name of the Bibliography in this case
% You must have a proper ".bib" file
%  and remember to run:
% latex bibtex latex latex
% to resolve all references
%
% ACM needs 'a single self-contained file'!
%
%APPENDICES are optional
%\balancecolumns
%\appendix
%Appendix A
%\section{Sample Tweets}

%\balancecolumns % GM June 2007
% That's all folks!
\end{document}
